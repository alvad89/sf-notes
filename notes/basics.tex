\documentclass[12pt,a4paper,draft]{article}
\usepackage[utf8x]{inputenc}
\usepackage[english,russian]{babel}
\usepackage{ucs}
\usepackage{amsmath}
\usepackage{amsfonts}
\usepackage{amssymb}
\usepackage{verbatim}
\begin{document}
\section{Enumerated Types}

В языке Coq практически нет никаких встроенных батареек -- ни логических значений, ни чисел. Coq предоставляет мощные стредства для определения новых типов данных и функций, которые оперируют и трансформируют эти типы.

\subsection{Days of the Week}
Рассмотрим простой пример:

\begin{verbatim}
Inductive day : Type :=
  | monday : day
  | tuesday : day
  | wednesday : day
  | thursday : day
  | friday : day
  | saturday : day
  | sunday : day.
\end{verbatim}

Это объявление говорит Coq, что мы определяем новый набор значений данных -- "тип". Этот тип назвали \texttt{day}, а значения этого типа -- \texttt{monday}, \texttt{tuesday} и т.д.

После того, как мы определили тип, мы можем построить функцию над значениями этого типа:

\begin{verbatim}
Definition next_weekday (d:day) : day :=
  match d with
  | monday => tuesday
  | tuesday => wednesday
  | wednesday => thursday
  | thursday => friday
  | friday => monday
  | saturday => monday
  | sunday => monday
  end.
\end{verbatim}

В Coq есть механизм вывода типов, который почти всегда может вывести тип, но для простоты понимания лучше всегда указывать сигнатуры.

После того, как мы определили функцию, самое время проверить ее работоспособность на нескольких примерах. В Coq есть три способа сделать это. Во-первых, мы можем использовать команду \texttt{Eval simpl}:

\begin{verbatim}
Eval simpl in (next_weekday friday).
   (* ==> monday : day *)
Eval simpl in (next_weekday (next_weekday saturday)).
   (* ==> tuesday : day *)
\end{verbatim}

Во-вторых, мы можем явно указать какой результат вы ожидаем:

\begin{verbatim}
Example test_next_weekday:
  (next_weekday (next_weekday saturday)) = tuesday.
\end{verbatim}

После этого мы можем поручить Coq проверить это утверждение:

\begin{verbatim}
Proof. simpl. reflexivity. Qed.
\end{verbatim}

Третий способ заключается в экстракции определения в какой-нибудь язык программирования (OCaml, Scheme или Haskell). Это очень крутая возможность т.к. на выходе получится полностью корректная (fully certified) программа. Это одна из основых фич системы Coq, ради которой все и разрабатывалось.

\subsection{Booleans}

Аналогично мы можем определить тип \texttt{bool} для логических значений:

\begin{verbatim}
Inductive bool : Type :=
  | true : bool
  | false : bool.
\end{verbatim}

В стандартной библиотеке Coq есть определения и леммы для логических значений (\texttt{Coq.Init.Datatypes}), но мы будем делать все руками в академических целях.

Построим несколько полезных функций над значениями типа \texttt{bool}:

\begin{verbatim}
Definition negb (b:bool) : bool :=
  match b with
  | true => false
  | false => true
  end.

Definition andb (b1:bool) (b2:bool) : bool :=
  match b1 with
  | true => b2
  | false => false
  end.

Definition orb (b1:bool) (b2:bool) : bool :=
  match b1 with
  | true => true
  | false => b2
  end.
\end{verbatim}

Мы можем определять и функции нескольких переменных. Теперь напишем 4 "юнит-теста", которые составляют полную спецификацию -- таблицу истинности -- для функции \texttt{orb}:

\begin{verbatim}
Example test_orb1: (orb true false) = true.
Proof. simpl. reflexivity. Qed.
Example test_orb2: (orb false false) = false.
Proof. simpl. reflexivity. Qed.
Example test_orb3: (orb false true) = true.
Proof. simpl. reflexivity. Qed.
Example test_orb4: (orb true true) = true.
Proof. simpl. reflexivity. Qed.
\end{verbatim}

\section{Function Types}
В Coq есть команда \texttt{Check}, которая печатает тип выражения.

\begin{verbatim}
Check true.
(* ===> true : bool *)
Check (negb true).
(* ===> negb true : bool *)
\end{verbatim}

Функции тоже имеют свой тип:

\begin{verbatim}
Check negb.
(* ===> negb : bool -> bool *)
\end{verbatim}

Из этой записи видно, что \texttt{negb} принимает на вход одно значение типа \texttt{bool} и возвращает значение типа \texttt{bool}. У функций нескольких переменных сигнатура будет иметь примерно такой вид:

\begin{verbatim}
Check andb.
(* ===> andb : bool -> bool -> bool *)
\end{verbatim}

Это одначает, что функция принимает на вход два аргумента типа \texttt{bool} и возвращает значение типа \texttt{bool}.
\end{document}